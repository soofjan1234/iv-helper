% !TEX TS-program = xelatex
% !TEX encoding = UTF-8 Unicode
% !Mode:: "TeX:UTF-8"

\documentclass{resume}
\usepackage{zh_CN-Adobefonts_external}
\usepackage{linespacing_fix}
\usepackage{cite}

\begin{document}
\pagenumbering{gobble}

\name{侯汶政|Go后端工程师}

\begin{center}
\textbf{微信:1489938120} | 
\textbf{网站:\href{https://Soofjan.com}{Soofjan.com}} |
\textbf{当前状态:深圳在职} 
\end{center}

\section{\faGraduationCap\ 教育背景}
\datedsubsection{\textbf{广州大学} \quad 学士,计算机科学与技术}{2019.09 -- 2023.06}
CET-6 英语六级(602分),全国大学生算法设计与编程挑战赛(银奖) 

\section{\faCogs\ 技术能力}
\begin{itemize}[parsep=0.3ex]
  \item \textbf{编程语言与框架}:使用 Go 及其生态工具 Gin、GORM 进行开发,深入理解其并发模型(Goroutine)、内存管理机制(GC);拥有 Vue 实战经验,具备多端适应能力。
  \item \textbf{数据存储与缓存}:掌握 MySQL 和 SQLite 的索引与优化;熟悉 Redis 的数据结构、持久化机制及缓存雪崩、穿透、击穿等问题的高效解决方案。
  \item \textbf{中间件}: 熟悉 NSQ 消息队列,能处理消息丢失/重复问题;了解Kafka、RabbitMQ等消息队列在不同场景下的使用选择。
  \item \textbf{开发工具}:使用 Git 协作,熟悉 Linux 环境、Shell 脚本编程;高效使用 Cursor、CC 等 AI 工具。
  
\end{itemize}

\section{\faSuitcase\ 工作经历}
\datedsubsection{\textbf{软件工程师 @ 慧为智能科技有限公司}}{2023.05 - 至今}

\paragraph*{私有云存储系统(NAS)开发}
% \mbox{} \\
% \textcolor{gray}{ \textbf{技术栈:} Go, Vue, C++, MySQL, SQLite, NSQ, AES+RSA, ElasticSearch, Docker  }

\begin{itemize}

    \item \textbf{云端}:
    \begin{itemize}
        \item 主导设计 NAT 穿透交互架构,规范客户端、服务端、云端及穿透端的通信流程,实现复杂网络下稳定高效访问。
        \item 引入 NSQ 消息队列,实现事件异步解耦,系统可稳定支持 QPS 3 万+ 的高并发写入,有效应对高并发场景下的系统压力。
        \item 采用 AES+RSA 混合加密方案保障数据传输安全,实现端到端加密通信。
    \end{itemize}

    \item \textbf{智能文档中心}:
    \begin{itemize}
        \item 针对内存受限环境,完成了从 Elasticsearch 到 Bleve 的搜索引擎迁移,结合GPT 语义扩展与高性能缓存,实现快速的全文与多维度检索。
        \item 针对超长文档,设计了基于块的索引策略与搜索结果聚合算法,为用户提供了具备高上下文关联性的“搜索高亮快照”。
        \item 支持 6 种文档格式解析,通过 chardet 解决跨编码乱码问题,实现全量 UTF-8 归一化。
    \end{itemize}

    \item \textbf{文件索引同步系统}:
    \begin{itemize}
        \item 从 fsnotify 事件监听模型切换至自研 DFS+Queue 架构,避免OOM风险,在海量文件场景下内存开销降低 90\% 以上。
        \item 实现基于 SQLite 状态机的增量同步引擎,通过文件指纹比对与任务优先级调度实现断点续传及冗余过滤,显著提升同步可靠性与响应效率。
        \item 基于接口抽象与 Functional Options 模式,实现同步逻辑与底层存储、IO 的深度解耦,支持跨项目的高内聚集成。
    \end{itemize}

    \item \textbf{AI相册}:
    \begin{itemize}
        \item 图片分类与人脸识别:通过 DBSCAN 聚类方法,支持 1 万人级人脸库的快速管理与比对,识别准确率达 97.3\%。
        \item 后端:在检索模块中,运用策略、工厂模式,实现独立模式与依赖网盘模式两种检索策略,满足不同部署场景需求。
    \end{itemize}

    \item \textbf{其他}:
    \begin{itemize}
        \item 开发并维护 Docker、虚拟机、Webdav、ChatGPT 等前后端项目,提供 NAS 生态支持。
        \item 通过 rate 限流、Ristretto 缓存库、sync.Map+sync.Mutex 的方式有效缓解并发流量冲击;核心接口缓存命中率达 85\%,响应时间缩短 120ms。
    \end{itemize}

\end{itemize}

\end{document}